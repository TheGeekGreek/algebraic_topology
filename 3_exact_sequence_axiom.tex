\section*{The Exact Sequence Axiom}

\subsection*{Homological Algebra}
\subsubsection*{Diagram Lemmas}
\begin{proposition}[Snake Lemma]
	Suppose we are given a commutative diagram in $\mathsf{AbGrp}$ with exact rows:
	\begin{equation*}
		\begin{tikzcd}
			& A \arrow[r,"i"]\arrow[d,"f"] & B \arrow[r,"j"]\arrow[d,"g"] & C \arrow[r]\arrow[d,"h"] & 0\\
			0 \arrow[r] & A' \arrow[r,"i'"'] & B' \arrow[r,"j'"'] & C'
		\end{tikzcd}
	\end{equation*}
	Then there exists $\delta \in \mathsf{AbGrp}(\ker h,\coker f)$ such that the sequence
	\begin{equation}
		\begin{tikzcd}
			 \ker f \arrow[r] & \ker g \arrow[r] & \ker h \arrow[r,"\delta"] & \coker f \arrow[r] & \coker g \arrow[r] & \coker h
		 \end{tikzcd}
	\end{equation}
	\noindent is exact.
	\label{prop:snake_lemma}
\end{proposition}

\begin{proof}
	Consider the augmented diagram in figure \ref{fig:snake_lemma}, where the morphisms $k$, $l$, $p$ and $q$ are induced by $i$, $j$, $i'$ and $j'$, respectively.
	\begin{figure}[h!tb]
		\centering
		\begin{tikzcd}
			& 0 \arrow[d] & 0 \arrow[d] & 0 \arrow[d]\\
			& \ker f \arrow[d]\arrow[r,"k"] & \ker g \arrow[d]\arrow[r,"l"] & \ker h \arrow[d]\\
			& A \arrow[r,"i"]\arrow[d,"f"] & B \arrow[r,"j"]\arrow[d,"g"] & C \arrow[r]\arrow[d,"h"] & 0\\
			0 \arrow[r] & A' \arrow[r,"i'"']\arrow[d] & B' \arrow[r,"j'"']\arrow[d] & C' \arrow[d]\\
			& \coker f \arrow[d]\arrow[r,"p"'] & \coker g \arrow[d]\arrow[r,"q"'] & \coker h \arrow[d]\\
			& 0 & 0 & 0
		\end{tikzcd}		
		\caption{Proof of the snake lemma.}
		\label{fig:snake_lemma}
	\end{figure}

	\begin{enumerate}[label = \textit{Step \arabic*:},wide = 0pt]
		\item \textit{Exactness at $\ker g$.} Let $a \in \ker f$. Then $l(k(a)) = j(i(a)) = 0$ by exactness at $B$ and thus $\im k \subseteq \ker l$. Conversly, let $b \in \ker l$. Then $j(b) = 0$ and by exactness at $B$, there exists $a \in A$ such that $i(a) = b$. Moreover $0 = g(b) = g(i(a)) = i'(f(a))$ since $b \in \ker g$ and thus $f(a) = 0$ by injectivity of $i'$. Hence $\ker j \subseteq \im k$.
		\item \textit{Exactness at $\coker g$.} Let $a' + \im f \in \coker f$. Then
			\begin{equation*}
				q(p(a' + \im f)) = j'(i'(a')) + \im h = \im h
			\end{equation*}
			\noindent by exactness at $B'$ implies $\im p \subseteq \ker q$. Conversly, let $b' + \im g \in \ker q$. Then 
			\begin{equation*}
				0 = q(b' + \im g) = j'(b') + \im h
			\end{equation*}
			\noindent and thus $j'(b') \in \im h$. Hence there exists $c \in C$, such that $j'(b') = h(c)$. Since $j$ is surjective, we find $b \in B$ such that $j(b) = c$. Therefore $j'(b') = h(j(b))$. By commutativity we get $j'(b') = j'(g(b))$ which is equivalent to $j'(b' - g(b)) = 0$. Thus $b' - g(b) \in \ker j'$ and exactness at $B'$ yields the existence of $a' \in A'$ such that $i'(a') = b' - g(b)$. Now
			\begin{equation*}
				p(a' + \im f) = i'(a') + \im g = b' - g(b) + \im g = b' + \im g
			\end{equation*}
			\noindent and thus $\ker q \subseteq \im p$.
		\item \textit{Definition of $\delta$.} Consider the snakelike path indicated in figure \ref{fig:snakelike_path}. Let $c \in \ker h$. Since $j$ is surjective, we find $b \in B$ such that $j(b) = c$. Since $c \in \ker h$, we get that $0 = h(c) = h(j(b)) = j'(g(b))$ and thus $g(b) \in \ker j'$ which implies $g(b) \in \im i'$ by exactness at $B'$. Hence there exists $a' \in A'$ such that $i'(a') = g(b)$. Actually this $a'$ is unique since $i'$ is injective. Define $\delta : \ker h \to \coker f$ by
			\begin{equation*}
				\delta(c) := a' + \im f.
			\end{equation*}
			\begin{figure}[h!tb]
				\centering
				\begin{subfigure}[b]{.5\textwidth}
					\begin{tikzcd}
						\ker f \arrow[r,"k"] & \ker g \arrow[r,"l"] & \ker h \arrow[d]\\
						& B \arrow[d,"g"]& C \arrow[l]\\
						A' \arrow[d] & B' \arrow[l]\\
						\coker f \arrow[r,"p"'] & \coker g \arrow[r,"q"'] & \coker h
					\end{tikzcd}		
					\caption{Snakelike path.}
					\label{fig:snakelike_path}
				\end{subfigure}
				~
				\begin{subfigure}[b]{.5\textwidth}
					\begin{tikzcd}
						& & c \arrow[d,mapsto]\\
						& b \arrow[d,mapsto]& c \arrow[l,mapsto]\\
						a' \arrow[d,mapsto] & g(b) \arrow[l,mapsto]\\
						a' + \im f
					\end{tikzcd}		
					\caption{Definition of $\delta$.}
					\label{fig:definition_delta}					
				\end{subfigure}
				\caption{}
			\end{figure}
		\item \textit{Checking that $\delta$ is a morphism of groups.} Since $j$ is only surjective, we have to show that $\delta$ is a function. So suppose we choose $b_0 \in B$ instead of $b \in B$ in figure \ref{fig:definition_delta} with $b_0 \neq b$. We want to show that $\delta(c) = a' + \im f = a_0' + \im f$, or equivalently $a' - a_0' \in \im f$. Since $c = j(b) = j(b_0)$, we have that $b - b_0 \in \ker j$. Hence by exactness at $B$ there exists $a \in A$ such that $i(a) = b - b_0$. Applying $g$ and invoking commutativity yields
			\begin{equation*}
				g(b) - g(b_0) = g(i(a)) = i'(f(a))
			\end{equation*}
			Hence $i'(a') - i'(a_0')= i'(f(a))$ and thus the injectivity of $i'$ yields $a' - a_0' = f(a)$. In the same manner one can show that $\delta$ is a morphism of groups.
		\item \textit{Exactness at $\ker h$.} Let $b \in \ker g$. Then $\im l \subseteq \ker \delta$ immeddiately follows from figure \ref{fig:im_l_subseteq_ker_delta}. Conversly, suppose $c \in \ker \delta$. From figure \ref{fig:ker_delta_subseteq_im_l} we get that
			\begin{equation*}
				g(b) = i'(a') = i'(f(a)) = g(i(a))
			\end{equation*}
			\noindent and thus $b - i(a) \in \ker g$. So $l(b - i(a)) = j(b) - j(i(a)) = j(b) = c$ by exactness at $B$ and thus $\ker \delta \subseteq \im l$.
			\begin{figure}[h!tb]
				\centering
				\begin{subfigure}[b]{.5\textwidth}
					\begin{tikzcd}
						& & j(b) \arrow[d,mapsto]\\
						& b \arrow[d,mapsto]& j(b) \arrow[l,mapsto]\\
						0 \arrow[d,mapsto] & g(b) = 0 \arrow[l,mapsto]\\
						\im f\\
					\end{tikzcd}		
					\caption{$\im l \subseteq \ker \delta$.}
					\label{fig:im_l_subseteq_ker_delta}
				\end{subfigure}
				~
				\begin{subfigure}[b]{.5\textwidth}
					\begin{tikzcd}
						& & c \arrow[d,mapsto]\\
						a \arrow[d,mapsto] & b \arrow[d,mapsto]& c \arrow[l,mapsto]\\
						a' = f(a) \arrow[d,mapsto] & g(b) \arrow[l,mapsto]\\
						a' + \im f\\
					\end{tikzcd}		
					\caption{$\ker \delta \subseteq \im l$.}
					\label{fig:ker_delta_subseteq_im_l}					
				\end{subfigure}
				\caption{}
			\end{figure}

		\item \textit{Exactness at $\coker f$.} Suppose that $a' + \im f \in \im \delta$. Then
			\begin{equation*}
				p(a' + \im f) = i'(a') + \im g = g(b) + \im g = \im g 
			\end{equation*}
			\noindent and thus $\im \delta \subseteq \ker p$. Conversly, suppose that $a' + \im f \in \ker p$. Hence $i'(a') \in \im g$ and we find $b \in B$ such that $g(b) = i'(a')$. Consider $j(b)$. By exactness at $B'$ follows
			\begin{equation*}
				h(j(b)) = j'(g(b)) = j'(i'(a')) = 0
			\end{equation*}
			So $j(b) \in \ker h$. Moreover, by construction $\delta(j(b)) = a' + \im f$ and thus $\ker p \subseteq \im \delta$.
	\end{enumerate}
\end{proof}

\begin{proposition}[Long Exact Sequence in Homology]
	\label{prop:les_homology}
	Let
	\begin{equation*}
		\begin{tikzcd}
			0 \arrow[r] & C_\bullet \arrow[r,"f_\bullet"] & C'_\bullet \arrow[r,"g_\bullet"] & C_\bullet'' \arrow[r] & 0
		\end{tikzcd}
	\end{equation*}
	\noindent be a short exact sequence in $\mathsf{Comp}$. Then there exists a sequence $(\delta_n)_{n \in \mathbb{Z}}$, where for all $n \in \mathbb{Z}$, $\delta_n \in \mathsf{AbGrp}(H_n(C_\bullet''),H_{n - 1}(C_\bullet))$ and such that
	\begin{equation*}
		\begin{tikzcd}[column sep=7ex]
			\cdots \arrow[r] & H_n(C_\bullet) \arrow[r,"H_n(f)"] & H_n(C'_\bullet) \arrow[r,"H_n(g)"] & H_n(C''_\bullet) \arrow[r,"\delta_n"] & H_{n - 1}(C_\bullet) \arrow[r] & \cdots
		\end{tikzcd} 
	\end{equation*}
	\noindent is a long exact sequence in $\mathsf{AbGrp}$.
\end{proposition}

\begin{proof}
	Let $n \in \mathbb{Z}$ and consider the following diagram of induced morphisms:
	\begin{equation}
		\label{eq:les_homology}
		\begin{tikzcd}
			& C_n/\im\partial_{n + 1} \arrow[r,"f_n"]\arrow[d,"\partial_n"] & C_n'/\im\partial'_{n + 1} \arrow[r,"g_n"]\arrow[d,"\partial'_n"] & C_n''/\im\partial_{n + 1}'' \arrow[r]\arrow[d,"\partial''_n"] & 0\\
			0 \arrow[r] & \ker \partial_{n - 1} \arrow[r,"f_{n - 1}"'] & \ker \partial_{n - 1}' \arrow[r,"g_{n - 1}"'] & \ker\partial_{n - 1}''
		\end{tikzcd}		
	\end{equation}
	It is left to the reader to show that the induced maps are actually well defined, the diagram commutes and the rows are exact. Hence an application of the snake lemma \ref{prop:snake_lemma} yields $\delta_n \in \mathsf{AbGrp}(\ker \partial_n'',\coker \partial_n)$ and an exact sequence
	\begin{equation*}
		\begin{tikzcd}
			\ker \partial_n \arrow[r,"f_n"] & \ker \partial_n' \arrow[r,"g_n"] & \ker \partial_n'' \arrow[r,"\delta_n"] & \coker \partial_n \arrow[r,"f_{n - 1}"] & \coker \partial_n' \arrow[r,"g_{n - 1}"] & \coker \partial_n''
		 \end{tikzcd}
	\end{equation*}
	It is easy to check that this exact sequence is the same as
	\begin{equation*}
		\begin{tikzcd}[column sep=3.5ex]
			H_n(C_\bullet) \arrow[r,"H_n(f)"] & H_n(C'_\bullet) \arrow[r,"H_n(g)"] & H_n(C''_\bullet) \arrow[r,"\delta_n"] & H_{n - 1}(C_\bullet) \arrow[r,"H_{n-1}(f)"] & H_{n - 1}(C_\bullet') \arrow[r,"H_{n-1}(g)"] & H_{n - 1}(C_\bullet'').
		 \end{tikzcd}
	\end{equation*}
\end{proof}

\begin{exercise}
	In the proof of theorem \ref{prop:les_homology} in the diagram, show that the induced maps are actually well defined, the diagram commutes and the two rows are exact.
\end{exercise}

\begin{definition}[Connecting Homomorphism]
	The sequence $(\delta_n)_{n \in \mathbb{Z}}$ of morphisms in $\mathsf{AbGrp}$ of theorem \ref{prop:les_homology} is called the \bld{connecting homomorphism of the short exact sequence $0 \to C_\bullet \to C_\bullet' \to C_\bullet'' \to 0$}.
\end{definition}

\begin{corollary}[The Exact Sequence Axiom]
	Consider the relative homology functors $(H_n)_{n \in \omega}$. Moreover, for each $(X,A) \in \ob(\mathsf{Top})^2$, let $(\delta_{n,(X,A)})_{n \in \omega}$ be the sequence of connecting homomorphisms of the short exact sequence 
	\begin{equation*}
		\begin{tikzcd}[column sep = 7ex]
			0 \arrow[r] & C_\bullet(A) \arrow[r,"C_\bullet(\iota_A)"] & C_\bullet(X) \arrow[r,"C_\bullet(\iota_X)"] & C_\bullet(X,A) \arrow[r] & 0,
		\end{tikzcd}
	\end{equation*}
	\noindent where $\iota_A : (A,\varnothing) \hookrightarrow (X,\varnothing)$ and $\iota_X : (X,\varnothing) \hookrightarrow (X,A)$ denote inclusions. Then $\delta_n$ is a natural transformation for $n > 0$ and there is a long exact sequence
\begin{equation*}
	\begin{tikzcd}[column sep=7ex]
	\cdots \arrow[r] & H_n(A) \arrow[r,"H_n(\iota_A)"] & H_n(X) \arrow[r,"H_n(\iota_X)"] & H_n(X,A) \arrow[r,"\delta_{n,(X,A)}"] & H_{n - 1}(A) \arrow[r] & \cdots
	\end{tikzcd}
\end{equation*}

\end{corollary}

\begin{proof}
	The only thing to show is that $\delta_n : H_n \Rightarrow H_{n - 1} \circ R$ is a natural transformation for $n > 0$. Hence for any $f \in \mathsf{Top}^2\del[1]{(X,A),(Y,B)}$, we have to show that
	\begin{equation*}
		\begin{tikzcd}[column sep=7ex]
			H_n(X,A) \arrow[r,"\delta_{n,(X,A)}"]\arrow[d,"H_n(f)"'] & H_{n - 1}(A)\arrow[d,"H_{n-1}(R(f))"]\\
			H_n(Y,B) \arrow[r,"\delta_{n,(Y,B)}"']& H_{n - 1}(B)
		\end{tikzcd}
	\end{equation*}
	\noindent commutes. To this end, consider the following commutative diagram with exact rows in $\mathsf{Comp}$: 
	\begin{equation*}
		\begin{tikzcd}[column sep=7ex]
			0 \arrow[r] & C_\bullet(A) \arrow[r,"C_\bullet(\iota_A)"]\arrow[d,"C_\bullet(R(f))"] & C_\bullet(X) \arrow[r,"C_\bullet(\iota_X)"]\arrow[d,"C_\bullet(S(f))"] & C_\bullet(X,A) \arrow[r]\arrow[d,"C_\bullet(f)"] & 0\\
			0 \arrow[r] & C_\bullet(\iota_B) \arrow[r,"C_\bullet(\iota_B)"'] & C_\bullet(Y) \arrow[r,"C_\bullet(\iota_Y)"'] & C_\bullet(Y,B) \arrow[r] & 0,
		\end{tikzcd}
	\end{equation*}
	\noindent where $S : \mathsf{Top}^2 \to \mathsf{Top}^2$ is th functor defined by $S(X,A) := (X,\varnothing)$. Applying proposition \ref{prop:naturality_connecting_homomorphism} to the above diagram yields the result.
\end{proof}
