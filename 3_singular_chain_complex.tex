\subsection*{The Singular Chain Complex}

\begin{proposition}
	\label{prop:free_module_functor}
	Let $R \in \mathsf{Ring}$. Then the forgetful functor $U : {_R}\mathsf{Mod} \to \mathsf{Set}$ admits a left adjoint.
\end{proposition}

\begin{proof}
	Consider the \bld{free module functor} $F: \mathsf{Set} \to {_R}\mathsf{Mod}$ defined as follows:
	\begin{enumerate}[label = \textit{Step \arabic*:},wide = 0pt]
		\item \textit{Definition on objects.} Let $S \in \mathsf{Set}$ and define 
			\begin{equation*}
				F(S) := \cbr[1]{f \in R^S : \supp f \text{ is finite}}.
			\end{equation*}
			Equipped with pointwise defined addition and multiplication, $F(S)$ is a left $R$-module. Moreover, there is an inclusion $\iota : S \hookrightarrow U(F(S))$ sending $x \in S$ to the function taking the value one at $x$ and zero else. It is easy to check that $F(S)$ is free on $S$.
		\item \textit{Definition on morphisms.} Let $f : S \to S'$ in $\mathsf{Set}$, define $F(f) : F(S) \to F(S')$ by setting 
			\begin{equation*}
				F(f)\del[4]{\sum_{x \in S}r_x x} := \sum_{x \in S}r_x f(x).
			\end{equation*}
		\item $F \dashv U$. Let $M \in {_R}\mathsf{Mod}$ and $\varphi \in {_R}\mathsf{Mod}(F(S), M)$. Define $\wbar{\varphi} \in {_R}\mathsf{Mod}(S,U(M))$ to be the restriction to $S$ of the underlying map of sets. Conversly, if $f \in \mathsf{Set}(S,U(G))$, \bld{extending by linearity} yields $\wbar{f} \in {_R}\mathsf{Mod}(F(S),M)$ given by
		\begin{equation*}
			\wbar{f}\del[4]{\sum_{x \in S} r_x x} := \sum_{x \in S} r_x f(x).
		\end{equation*}
		It is now easy to check that $\wbar{\wbar{\varphi}} = \varphi$ and $\wbar{\wbar{f}} = f$ holds. 
	\end{enumerate}	
\end{proof}

\begin{exercise}
	In the proof of proposition \ref{prop:free_module_functor}, check functoriality of $F$ and naturality of the bijection ${_R}\mathsf{Mod}(F(S),M) \cong \mathsf{Set}(S,U(M))$.
\end{exercise}

\begin{theorem}[Singular Chain Complex Functor]
	\label{thm:singular_chain_complex_functor}
	Let $R \in \mathsf{Ring}$. Then there exists a functor 
	\begin{equation*}
		C_\bullet : \mathsf{Top} \to \Ch_{\geq 0}({_R}\mathsf{Mod}).
	\end{equation*}
\end{theorem}

\begin{proof}
	The proof is divided into two steps.
	\begin{enumerate}[label = \textit{Step \arabic*:},wide = 0pt]
		\item \textit{Definition on objects.} Let $X \in \mathsf{Top}$. Then define 
			\begin{equation*}
				C_\bullet(X)_n := F(\mathsf{Top}(\Delta^n,X))
			\end{equation*}
			\noindent for all $n \in \omega$, where $F : \mathsf{Set} \to {_R}\mathsf{Mod}$ denotes the free module functor from proposition \ref{prop:free_module_functor} and $\Delta^n$ denotes the $n$-th standard simplex from example \ref{ex:standard_simplex}.\\
			Let $\sigma \in \mathsf{Top}(\Delta^n,X)$, $n \geq 1$. Define
			\begin{equation}
				\label{eq:boundary_map}
				\partial_n \sigma := \sum_{k = 0}^n (-1)^k \sigma \circ \varphi^n_k,
			\end{equation}
			\noindent where $\varphi^n_k$ denotes the $k$-th face map in dimension $n$ from example \ref{ex:face_map}. Extending by linearity yields a morphism of $R$-modules $\partial_n : C_\bullet(X)_n \to C_\bullet(X)_{n - 1}$. For any $\sigma \in \mathsf{Top}(\Delta^{n + 1},X)$ we compute
			\begin{align*}
				(\partial_n \circ \partial_{n + 1})(\sigma) &= \partial_n \del[4]{\sum_{k = 0}^{n + 1} (-1)^k \sigma \circ \varphi^{n + 1}_k}\\
				&= \sum_{k = 0}^{n + 1} (-1)^k \partial_n\del[1]{\sigma \circ \varphi^{n + 1}_k}\\
				&= \sum_{k = 0}^{n + 1}\sum_{j = 0}^n (-1)^{k + j} \sigma \circ \varphi^{n + 1}_k \circ \varphi^n_j\\
				&= \sum_{0 \leq k \leq j \leq n}(-1)^{k + j} \sigma \circ \varphi^{n + 1}_k \circ \varphi^n_j + \sum_{0 \leq j < k \leq n + 1}(-1)^{k + j} \sigma \circ \varphi^{n + 1}_k \circ \varphi^n_j\\
				&= \sum_{0 \leq j \leq k \leq n}(-1)^{k + j} \sigma \circ \varphi^{n + 1}_j \circ \varphi^n_k + \sum_{0 \leq j <k \leq n + 1}(-1)^{k + j} \sigma \circ \varphi^{n + 1}_k \circ \varphi^n_j\\
				&= \sum_{0 \leq j < k \leq n + 1}\del[1]{(-1)^{k + j - 1} \sigma \circ \varphi^{n + 1}_j \circ \varphi^n_{k - 1} + (-1)^{k + j} \sigma \circ \varphi^{n + 1}_k \circ \varphi^n_j}
			\end{align*}
			Since $\varphi_j^{n + 1} \circ \varphi_{k - 1}^n$ and $\varphi_k^{n + 1} \circ \varphi_j^n$ are both equal to $A(e_0,\dots,\what{e_j},\dots,\what{e_k},\dots,e_{n + 1})$, it follows that
			\begin{equation*}
				\partial_n \circ \partial_{n + 1} = 0.
			\end{equation*}
			Indeed, consider the following chart of vertex maps:
			\begin{equation*}
				\begin{matrix}
					& \varphi_{k - 1}^n & & \varphi_j^{n + 1}\\
					e_0 & \mapsto & e_0 & \mapsto & e_0\\
					\vdots & & \vdots & & \vdots\\
					e_j & \mapsto & e_j & \mapsto & e_{j + 1}\\
					\vdots & & \vdots & & \vdots\\
					e_{k - 1} & \mapsto & e_k & \mapsto & e_{k + 1}\\
					\vdots & & \vdots & & \vdots\\
					e_{n - 1} & \mapsto & e_n & \mapsto & e_{n + 1}
				\end{matrix}
				\qquad\qquad\qquad
				\begin{matrix}
					& \varphi_j^n & & \varphi_k^{n + 1}\\
					e_0 & \mapsto & e_0 & \mapsto & e_0\\
					\vdots & & \vdots & & \vdots\\
					e_j & \mapsto & e_{j + 1} & \mapsto & e_{j + 1}\\
					\vdots & & \vdots & & \vdots\\
					e_{k - 1} & \mapsto & e_k & \mapsto & e_{k + 1}\\
					\vdots & & \vdots & & \vdots\\
					e_{n - 1} & \mapsto & e_n & \mapsto & e_{n + 1}
				\end{matrix}.
			\end{equation*}
		\item \textit{Definition on morphisms.} Let $f : X \to Y$ be a morphism in $\mathsf{Top}$. For $n \in \omega$, define $C_\bullet(f)_n : C_\bullet(X)_n \to C_\bullet(Y)_n$ by
			\begin{equation*}
				C_\bullet(f)_n(\sigma) := f \circ \sigma,
			\end{equation*}
			\noindent for any $\sigma \in \mathsf{Top}(\Delta^n,X)$. We compute 
			\begin{equation*}
				(\partial_n \circ C_\bullet(f)_n)(\sigma) = \sum_{k = 0}^n (-1)^k f \circ \sigma \circ \varphi^n_k = (C_\bullet(f)_{n - 1}\circ \partial_n)(\sigma).
			\end{equation*}
			\noindent Thus $C_\bullet(f)$ is a morphism in $\Ch_{\geq 0}({_R}\mathsf{Mod})$.
	\end{enumerate}
	Checking functoriality is left as an exercise.
\end{proof}

\begin{exercise}
	Check that $C_\bullet : \mathsf{Top} \to \Ch_{\geq 0}({_R}\mathsf{Mod})$ defined in theorem \ref{thm:singular_chain_complex_functor} is indeed a functor.
\end{exercise}

\begin{theorem}[Relative Singular Chain Complex Functor]
	\label{thm:relative_singular_chain_complex_functor}
	Let $R \in \mathsf{Ring}$. Then there exists a functor 
	\begin{equation*}
		C_\bullet : \mathsf{Top}^2 \to \Ch_{\geq 0}({_R}\mathsf{Mod}).
	\end{equation*}
\end{theorem}

\begin{proof}
	
\end{proof}
