\chapter{Basic Point-Set Topology}
\section*{The Category of Topological Spaces}
\subsection*{Topologies}
We follow \cite[20]{lee:topological_manifolds:2011}.

\begin{definition}[Topology]
	Let $X$ be a set. A \bld{topology on $X$} is a collection $\mathcal{T} \subseteq 2^X$ of subsets of $X$, satisfying the following properties:
	\begin{enumerate}[label = \textup{(}\roman*\textup{)},wide = 0pt]
		\item $X,\varnothing \in \mathcal{T}$.
		\item If $U_1,\dots,U_n \in \mathcal{T}$, then $U_1 \cap \dots \cap U_n \in \mathcal{T}$.
		\item If $(U_\alpha)_{\alpha \in A}$ is a family in $\mathcal{T}$ for an arbitrary set $A$, then $\bigcup_{\alpha \in A} U_\alpha \in \mathcal{T}$.
	\end{enumerate}
\end{definition}

\begin{definition}[Topological Space]
	A \bld{topological space} is a tuple $(X,\mathcal{T})$, where $X$ is a set and $\mathcal{T}$ is a topology on $X$.
\end{definition}

\begin{remark}
	In practice, the topology $\mathcal{T}$ on $X$ in the topologial space $(X,\mathcal{T})$ is often omitted for shortness.	
\end{remark}

\begin{definition}[Point]
	Let $(X,\mathcal{T})$ be a topological space. The elements of $X$ are called \bld{points}.
\end{definition}

\begin{definition}[Open and Closed Subsets]
	Let $(X,\mathcal{T})$ be a topologial space. A subset $U \in \mathcal{T}$ is called an \bld{open subset of $X$}. A subset $A \subseteq X$ is called a \bld{closed subset of $X$}, if its complement $X \setminus A$ is an open subset of $X$. 
\end{definition}

\begin{definition}[Neighbourhood]
	Let $X$ be a topological space and $A \subseteq X$ a subset. A \bld{neighbourhood of $A$ in $X$} is an open subset $U$ such that $A \subseteq U$. In particular, for some point $p$ in $X$, a neighbourhood of $\cbr{p}$ in $X$ is simply called a \bld{neighbourhood of $p$}.
\end{definition}

\subsection*{Continuity}

\begin{definition}[Continuity]
	Let $(X,\mathcal{T}_X)$ and $(Y,\mathcal{T}_Y)$ be topologial spaces. A function $f : X \to Y$ is called \bld{continuous}, if for all $U \in \mathcal{T}_Y$, $f^{-1}(U) \in \mathcal{T}_X$.  
\end{definition}

\subsection*{The Category $\mathsf{Top}$}

\begin{proposition}
	There exists a locally small category with objects topologial spaces and continuous maps as morphisms.
	\label{prop:Top}
\end{proposition}

\begin{exercise}
	Prove proposition \ref{prop:Top}.
\end{exercise}

\begin{definition}[$\mathsf{Top}$]
	The category defined in the proof of proposition \ref{prop:Top} is called the \bld{category of topologial spaces}, denoted by $\mathsf{Top}$.
\end{definition}

\section*{}<++>
\subsection*{The Lebesgue Number Lemma}

\begin{definition}[Lebesgue Number]
	Let $(M,d)$ be a metric space with an open cover $(U_\alpha)_{\alpha \in A}$. A number $\delta > 0$ is called a \bld{Lebesgue number} for the cover, if every subset of $M$ whose diameter is less than $\delta$ is contained in $U_\alpha$ for some $\alpha \in A$.
\end{definition}

\begin{lemma}[Lebesgue Number Lemma]
	\label{lem:Lebesgue_number_lemma}
	Every open cover of a compact metric space admits a Lebesgue number.	
\end{lemma}

\subsection*{The Closed Map Lemma}

\begin{lemma}[Closed Map Lemma]
	Let $X,Y \in \ob(\mathsf{Top})$ such that $X$ is compact and $Y$ is Hausdorff, and $f \in \mathsf{Top}(X,Y)$. Then:
	\begin{enumerate}[label = \textup{(}\alph*\textup{)},wide = 0pt]
		\item $f$ is a closed map.
		\item If $f$ is injective, it is a topological embedding.
		\item If $f$ is surjective, it is a quotient map.
		\item If $f$ is bijective, it is a homeomorphism.
	\end{enumerate}
	\label{lem:closed_map_lemma}
\end{lemma}


