\subsection*{The Singular Homology Functor}

\begin{proposition}
	\label{prop:tensor_chain_complex_fixed_module}
	Let $R \in \mathsf{CRing}$ and $M \in {_R}\mathsf{Mod}$. Then there exists an additive functor 
	\begin{equation*}
		(-)\otimes_R M : \Ch({_R}\mathsf{Mod}) \to \Ch({_R}\mathsf{Mod}).
	\end{equation*}
\end{proposition}

\begin{proof}
	
\end{proof}

\begin{definition}[Singular Homology Functor]
	Let $R \in \mathsf{CRing}$ and $M \in {_R}\mathsf{Mod}$. For $n \in \mathbb{Z}$, define the \bld{$n$-th singular homology functor} to be the composition
	\begin{equation}
		H_n \circ (-) \otimes_R M \circ C_\bullet : \mathsf{Top}^2 \to {_R}\mathsf{Mod},
	\end{equation}
	\noindent where $C_\bullet : \mathsf{Top}^2 \to \Ch_{\geq 0}({_R}\mathsf{Mod})$ denotes the relative singular chain complex functor defined in theorem \ref{thm:relative_singular_chain_complex_functor}, $(-) \otimes_R M : \Ch({_R}\mathsf{Mod}) \to \Ch({_R}\mathsf{Mod})$ denotes the functor defined in proposition \ref{prop:tensor_chain_complex_fixed_module} and $H_n : \Ch({_R}\mathsf{Mod}) \to {_R}\mathsf{Mod}$ denotes the usual $n$-th homology functor.
\end{definition}
