\chapter{Cohomology}

The archetypical example of a cohomology theory arieses in differential topology: The \emph{de Rham cohomology}. It is the homology of the non-negative cochain complex 
\begin{equation*}
	\begin{tikzcd}
		0 \arrow[r] & \Omega^0(M) \arrow[r,"d^0"] & \Omega^1(M) \arrow[r,"d^1"] & \dots
	\end{tikzcd}
\end{equation*}
\noindent where $M$ is a smooth manifold, $\Omega^k(M) := \Gamma(\Lambda^k T^\vee M)$ denotes the vector space of \emph{smooth differential $k$-forms} and $d : \Omega^k(M) \to \Omega^{k + 1}(M)$ denotes the \emph{exterior differentiation operators}. These extend the notion of a differential of a function and hence provide a more intuitive approach to cohomology than the mere algebraic one. For more on this topic, see for example \cite{lee:smooth_manifolds:2013}.

%The Cohomology Ring
\section*{The Cohomology Ring}
\subsection*{The Cup Product}

\begin{proposition}
	Let $X \in \ob(\mathsf{Top})$ and $R \in \ob(\mathsf{Ring})$. Then there exists a contravariant functor
	\begin{equation*}
		C(-;R) : \mathsf{Top} \to \mathsf{GRing}.
	\end{equation*}
\end{proposition}

\begin{proof}
	~
	\begin{enumerate}[label = \textit{Step \arabic*:},wide = 0pt]
		\item \textit{Definition on objects.} Let $X \in \mathsf{Top}$. For $\alpha \in C^n(X;R)$
			and $\beta \in C^m(X;R)$ define
			\begin{equation*}
				(\alpha \cupprod \beta)(\sigma) := \alpha(\sigma \circ A(e_0,\dots,e_n)) \beta(\sigma \circ A(e_n,\dots,e_{n + m})),
			\end{equation*}
			\noindent for all singular $n + m$-simplices $\sigma$ in $X$. Hence extending by linearity yields a map
			\begin{equation*}
				\cupprod : C^n(X;R) \times C^m(X;R) \to C^{n + m}(X;R).
			\end{equation*}
			Moreover, if 
			\begin{equation*}
				C(X;R) := \bigoplus_{n \in \omega } C^n(X;R),
			\end{equation*}
			\noindent we define $\cupprod : C(X;R) \times C(X;R) \to C(X;R)$ by
			\begin{equation*}
				\sum_i \alpha_i \cupprod \sum_j \beta_j := \sum_{i,j} \alpha_i \cupprod \beta_j.
			\end{equation*}
			This is called the \bld{cup product on $C(X;R)$}. It is easily verified that $(C(X;R),\cupprod) \in \mathsf{GRing}$.
		\item \textit{Definition on morphisms.} Let $n \in \omega$ and $f \in \mathsf{Top}(X,Y)$. For $\alpha \in C^n(Y;R)$ define
			\begin{equation*}
				C(f;R)(\alpha) := C^n(f)(\alpha) \in C^n(X;R),
			\end{equation*}
			\noindent and extend by linearity.
	\end{enumerate}
\end{proof}

