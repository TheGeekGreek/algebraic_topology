\section*{The Excision Axiom}
\subsection*{Barycentric Subdivision}

\begin{definition}[Cone]
	Let $m,n \in \omega$, $K \subseteq \mathbb{R}^m$ convex, $v_0,\dots,v_n,w \in K$ and suppose $\alpha := A(v_0,\dots,v_n)$ is an affine $n$-simplex. Then define the \bld{cone on $\alpha$ from $w$}, written $\Cone_w(\alpha)$, by
	\begin{equation*}
		\Cone_w(\alpha) := A(w,v_0,\dots,v_n).
	\end{equation*}
	Moreover, for an affine chain $c := \sum_{i}m_i \alpha_i$, define
	\begin{equation*}
		\Cone_w(c) := \sum_i m_i \Cone_w(\alpha_i).
	\end{equation*}
\end{definition}

\begin{lemma}
	\label{lem:partial_cone}
	Let $m,n \in \omega$, $K \subseteq \mathbb{R}^m$ convex, $v_0,\dots,v_n,w \in K$ and $\alpha := A(v_0,\dots,v_n)$. Then
	\begin{equation*}
		\partial \Cone_w(\alpha) + \Cone_w(\partial \alpha) = \alpha.
	\end{equation*}
\end{lemma}

\begin{proof}
	We compute
	\begin{align*}
		\Cone_w(\partial \alpha) &= \sum_{k = 0}^n (-1)^k \Cone_w\del[1]{A(v_0,\dots,\what{v}_k,\dots,v_n)}\\
		&= \sum_{k = 0}^n (-1)^kA(w,v_0,\dots,\what{v}_k,\dots,v_n),
	\end{align*}
	\noindent since $\alpha \circ \varphi_k^n = A(v_0,\dots,\what{v}_k,\dots,v_n)$. Thus
	\begin{align*}
		\partial \Cone_w(\alpha) &= \sum_{k = 0}^{n + 1} (-1)^k \Cone_w(\alpha) \circ \varphi_k^{n+1}\\
		&= \alpha + \sum_{k = 1}^{n + 1} (-1)^k A(w,v_0,\dots,\what{v}_{k - 1},\dots,v_n)\\
		&= \alpha - \sum_{k = 0}^n (-1)^k A(w,v_0,\dots,\what{v}_k,\dots,v_n)\\
		&= \alpha - \Cone_w(\partial \alpha).
	\end{align*}
\end{proof}

\begin{definition}[Barycenter]
	Let $\sigma := \sbr[0]{v_0,\dots,v_k} \subseteq \mathbb{R}^n$. Define the \bld{barycenter of $\sigma$}, written $b_\sigma$, to be
	\begin{equation*}
		b_\sigma := \frac{1}{1 + k}\sum_{i = 0}^k v_i.
	\end{equation*}
\end{definition}

\begin{definition}[Affine Barycentric Subdivision Operator]
	Let $K \subseteq \mathbb{R}^m$ be convex. Define the \bld{affine barycentric subdivision operator} $s : C_n^{\mathrm{aff}}(K) \to C^{\mathrm{aff}}_n(K)$ inductively by
	\begin{equation*}
		s^{\mathrm{aff}}_n(\sigma) := \ccases{
			\sigma & n = 0,\\
			\Cone_{\sigma(b_n)}\del[1]{s^{\mathrm{aff}}_{n - 1}(\partial\sigma)}, & n > 0,
		}
	\end{equation*}
	\noindent where $b_n := b_{\Delta^n}$, on affine $n$-simplices $\sigma : \Delta^n \to K$ and extend by linearity.
\end{definition}

\begin{exercise}
	\label{ex:formula_affine_barycentric_subdivision}
	Let $m,n \in \omega$, $K \subseteq \mathbb{R}^m$ convex, $v_0,\dots,v_n,w \in K$. Show that
	\begin{equation*}
		s^{\mathrm{aff}}_n(A(v_0,\dots,v_n)) = \sum_{\sigma \in S_{n + 1}} \sgn(\sigma) A(v_0^\sigma,\dots,v_n^\sigma),
	\end{equation*}
	\noindent where
	\begin{equation*}
		v_i^\sigma := \frac{1}{n - i + 1} \sum_{k = i}^n v_{\sigma(k)}.
	\end{equation*}
\end{exercise}

\begin{definition}[Barycentric Subdivision Operator]
	Let $X \in \ob(\mathsf{Top})$. Define the \bld{barycentric subdivision operator $s : C_n(X) \to C_n(X)$} by
	\begin{equation*}
		s_n(\sigma) := \ccases{
			0 & n < 0,\\	
			C_n(\sigma)\del[1]{s^{\mathrm{aff}}_n(\id_{\Delta^n})} & n \geq 0,
		}
	\end{equation*}
	\noindent on $n$-simplices $\sigma$.
\end{definition}

\begin{lemma}
	\label{lem:s_aff_s}
	Let $K \subseteq \mathbb{R}^m$ convex. Then for any affine $n$-simplex $\alpha$ we have that
	\begin{equation*}
		s_n(\alpha) = s^{\mathrm{aff}}_n(\alpha).
	\end{equation*}
\end{lemma}

\begin{proof}
	Let $\alpha := A(v_0,\dots,v_n)$. Observe that exercise \ref{ex:formula_affine_barycentric_subdivision} yields
	\begin{align*}
		s(\alpha) &= C_n(\alpha)(s^{\mathrm{aff}}(\id_{\Delta^n}))\\
		&= C_n(\alpha)\del[1]{s^{\mathrm{aff}}(A(e_0,\dots,e_n))}\\
		&= \sum_{\sigma \in S_{n + 1}} \sgn(\sigma) \alpha \circ A(e_0^\sigma,\dots,e_n^\sigma)\\
		&= \sum_{\sigma \in S_{n + 1}} \sgn(\sigma) A(v_0^\sigma,\dots,v_n^\sigma)\\
		&= s^{\mathrm{aff}}(\alpha).
	\end{align*}	
\end{proof}

\begin{lemma}
	Let $\sigma := \sbr[0]{v_0,\dots,v_n} \subseteq \mathbb{R}^m$ be some $n$-simplex. Then for all $x,y \in \sigma$ we have that
	\begin{equation*}
		\abs[0]{x - y} \leq \sup_{k = 0,\dots,n} \abs[0]{v_k - y}.
	\end{equation*}
	Moreover
	\begin{equation*}
		\abs[0]{b_\sigma - v_k} \leq \frac{n}{n + 1}\diam \sigma = \max_{i,j = 0,\dots,n}\abs[0]{v_i - v_j},
	\end{equation*}
	\noindent for all $k = 0,\dots,n$.
\end{lemma}

\begin{proof}
	Let $x,y \in \sigma$ and write $x = \sum_{k = 0}^n s_k v_k$. Since $\sum_{k = 0}^n s_k = 1$, we have that
	\begin{equation*}
		\abs[0]{x - y} = \abs[4]{\sum_{k = 0}^n s_k v_k - y} \leq \sum_{k = 0}^n s_k \abs[0]{v_k - y} \leq \sup_{k = 0,\dots,n} \abs[0]{v_k - y}.
	\end{equation*}
\end{proof}

\begin{definition}[Mesh]
	Let $K \subseteq \mathbb{R}^m$ convex and $c \in C_n^\mathrm{aff}(K)$. Define the \bld{mesh of $c$} to be the maximum of the diameters of the images of the affine simplices that appear in $c$.
\end{definition}

\begin{proposition}
	\label{prop:properties_s_n}
	The barycentric subdivision operators $(s_n)_{n \in \mathbb{Z}}$ have the following properties:
	\begin{enumerate}[label = \textup{(}\alph*\textup{)},wide = 0pt]
		\item For any $f \in \mathsf{Top}(X,Y)$, we have that $s_n \circ C_n(f) = C_n(f) \circ s_n$.
		\item $\partial_n \circ s_n = s_{n - 1} \circ \partial_n$.
		\item Given an open cover $\mathcal{U}$ of $X$ and some $c \in C_n(X)$, there exists $k \in \omega$, such that $s_n^k(c) \in C_n^\mathcal{U}(X)$.
	\end{enumerate}
\end{proposition}

\begin{proof}
	For proving (a), let $\sigma : \Delta^n \to X$ be an $n$-simplex. Then by the functoriality of $C_n$ we get that
	\begin{align*}
		\del[0]{s_n \circ C_n(f)}(\sigma) &= s_n(f \circ \sigma)\\
		&= C_n(f \circ \sigma)\del[0]{s_n^{\mathrm{aff}}(\id_{\Delta^n})}\\ 
		&= \del[0]{C_n(f) \circ C_n(\sigma)}(s^{\mathrm{aff}}_n(\id_{\Delta^n})\\ 
		&= (C_n(f) \circ s_n)(\sigma).
	\end{align*}
	For proving (b), we do an induction on $n$. For $n = 0$, this is trivially true. Hence assume (b) holds for some $n \in \omega$. Using lemma \ref{lem:partial_cone}, lemma \ref{lem:s_aff_s} and part (a) we compute
	\begin{align*}
		(\partial \circ s)(\sigma) &= \partial C_{n + 1}(\sigma) \del[1]{s^{\mathrm{aff}}(\id_{\Delta^{n+1}})}\\
		&= C_n(\sigma)\partial\Cone_{b_{n + 1}}\del[1]{s^{\mathrm{aff}}(\partial\id_{\Delta^{n+1}})}\\
		&= C_n(\sigma)\del[1]{s^{\mathrm{aff}}(\partial\id_{\Delta^{n+1}}) - \Cone_{b_{n + 1}}(\partial s^{\mathrm{aff}}\partial\id_{\Delta^{n+1}})}\\
		&= C_n(\sigma)\del[1]{s(\partial\id_{\Delta^{n+1}}) - \Cone_{b_{n + 1}}(\partial s\partial\id_{\Delta^{n+1}})}\\
		&= C_n(\sigma)\del[1]{s(\partial\id_{\Delta^{n+1}}) - \Cone_{b_{n + 1}}(s\partial^2\id_{\Delta^{n+1}})}\\
		&= C_n(\sigma)s(\partial\id_{\Delta^{n+1}})\\
		&= sC_n(\sigma)(\partial\id_{\Delta^{n+1}})\\
		&= s \partial C_{n + 1}(\sigma)(\id_{\Delta^{n + 1}})\\
		&= (s \circ \partial)(\sigma),
	\end{align*}
	\noindent for any $n + 1$-simplex $\sigma$.\\
	For proving (c), let $\sigma : \Delta^n \to X$ be any singular $n$-simplex. Then $\sigma^{-1}(\mathcal{U})$ is an open cover for $\Delta^n$ and hence there exists a Lebesgue number $\delta > 0$. 
\end{proof}

\begin{theorem}
	For each $n \in \mathbb{Z}$, $H_n(s_n) = \id_{H_n}$.
\end{theorem}


\begin{theorem}
	\label{thm:inclusion_U_small}
	Let $\mathcal{U}$ be an open cover for a topologial space $X$. Then the inclusion $\iota_\bullet : C_\bullet^\mathcal{U}(X) \hookrightarrow C_\bullet(X)$ induces an isomorphism $H_n^\mathcal{U}(X) \to H_n(X)$.
\end{theorem}

\begin{proof}
	\begin{enumerate}[label = \textit{Step \arabic*:},wide = 0pt]
		\item \textit{Construction of a chain homotopy between $s_n$ and $\id$.} We define inductively $F_n : C_n(X) \to C_{n + 1}(X)$ by
			\begin{equation*}
				F_n\sigma := \ccases{
					0 & n = 0,\\
					C_{n + 1}(\sigma)\Cone_{b_n}(\id_{\Delta^n} - s_n\id_{\Delta^n} - F_{n - 1}\partial_n\id_{\Delta^n}) & n > 0.
				}
			\end{equation*}
			We have to show now that
			\begin{equation*}
				\partial_{n + 1} \circ F_n + F_{n - 1} \circ \partial_n = \id - s_n
			\end{equation*}
			\noindent holds. We proceed by induction over $n \in \omega$. The case $n = 0$ is clear. Let us suppose that above identity holds for some $n - 1 \in \omega$. Then proposition \ref{prop:properties_s_n} and the induction hypothesis yields 
			\begin{align*}
				\partial F_n\sigma =& \partial C_{n + 1}(\sigma)\Cone_{b_n}(\id_{\Delta^n} - s_n\id_{\Delta^n} - F_{n - 1}\partial \id_{\Delta^n})\\
				=& C_n(\sigma)\partial\Cone_{b_n}(\id_{\Delta^n} - s_n\id_{\Delta^n} - F_{n - 1}\partial\id_{\Delta^n})\\
				=& \sigma - s_n\sigma - C_n(\sigma)F_{n - 1}\partial\id_{\Delta^n}\\
				&- C_n(\sigma)\Cone_{b_n}\del[1]{\partial\id_{\Delta^n} - \partial s_n\id_{\Delta^n} - \partial F_{n - 1}\partial_n\id_{\Delta^n}}\\
				=& \sigma - s_n\sigma - C_n(\sigma)F_{n - 1}\partial\id_{\Delta^n}\\
				&- C_n(\sigma)\Cone_{b_n}\del[1]{\partial\id_{\Delta^n} - s_{n - 1}\partial\id_{\Delta^n} - \partial F_{n - 1}\partial_n\id_{\Delta^n} - F_{n - 2}\partial\partial\id_{\Delta^n}}\\
				=& \sigma - s_n\sigma - C_n(\sigma)F_{n - 1}\partial\id_{\Delta^n}\\
				=& \sigma - s_n\sigma - F_{n - 1}\partial \sigma,
			\end{align*}
			\noindent since for any continuous map $f : X \to X$ it is easy to show that 
			\begin{equation*}
				C_n(f) \circ F_{n - 1} = F_{n - 1} \circ C_{n - 1}(f).
			\end{equation*}
		\item \textit{Injectivity.} Let $\langle c \rangle \in \ker H_n^\mathcal{U}(X)$. Hence there exists some $b \in C_{n + 1}(X)$ such that $\partial b = c$. By part (c) of proposition \ref{prop:properties_s_n} we find $k \in \omega$ such that $s^k(b) \in C_{n + 1}^\mathcal{U}(X)$. Moreover, $\partial s^k b = s^k \partial b = s^kc$. Now using the chain homotopy of the first step and induction, one can show that $s^kc$ and $c$ differ by a boundary and hence that $\langle s^k c \rangle = \langle c \rangle$. Thus $\langle c \rangle = \langle \partial s^k b \rangle = 0$.
		\item \textit{Surjectivity.} Let $\langle c \rangle \in H_n(X)$. Then by part (c) of proposition \ref{prop:properties_s_n} we find $k \in \omega$ such that $s^k c \in C_n^\mathcal{U}(X)$. Moreover, $\partial s^k c = s^k \partial c = 0$. Hence by the previous reasoning we have that $\langle s^kc \rangle = \langle c \rangle$.
	\end{enumerate}
\end{proof}

\subsection*{The Excision Axiom}

\begin{proposition}
	Let $(X,A) \in \ob(\mathsf{Top}^2)$ and $\mathcal{U}$ an open cover for $X$. Then the inclusion $C_\bullet^\mathcal{U}(X,Y) \hookrightarrow C_\bullet(X,A)$, where $C_\bullet^\mathcal{U}(X,A) := C_\bullet^\mathcal{U}(X)/C_\bullet^{\mathcal{U} \cap A}(A)$, induces an ismorphism $H_n^\mathcal{U}(X,A) \cong H_n(X,A)$. 
\end{proposition}

\begin{theorem}[The Excision Axiom]
	Let $U$ and $V$ be an open cover for a topological space $X$. Then the inclusion $(U,U \cap V) \hookrightarrow (X,V)$ induces an isomorphism in homology $H_n(U,U \cap V) \cong H_n(X,V)$
\end{theorem}

\begin{proof}
	Let $\mathcal{U} := \cbr[0]{U,V}$. Then we have that $C_\bullet^\mathcal{U}(X) = C_\bullet(U) + C_\bullet(V)$. Consider the short exact sequence
	\begin{equation*}
		\begin{tikzcd}
			0 \arrow[r] & C_\bullet(U) + C_\bullet(V) \arrow[r,"\iota_\bullet"] & C_\bullet(X) \arrow[r] & C_\bullet(X)/\del[0]{C_\bullet(U) + C_\bullet(V)} \arrow[r] & 0
		\end{tikzcd}
	\end{equation*}
	\noindent in $\mathsf{Comp}$. Using proposition \ref{prop:les_homology} we get a long exact sequence in homology and theorem \ref{thm:inclusion_U_small} implies that $H_n(\iota_\bullet)$ is an isomorphism. Since this is every third map in the long exact sequence, it is easy to show that 
	\begin{equation*}
		H_n\del[1]{C_\bullet(X)/(C_\bullet(U) + C_\bullet(V))} = 0.
	\end{equation*}
	Moreover, let us consider the short exact sequence
	\begin{equation*}
		\begin{tikzcd}
			0 \arrow[r] & \displaystyle\frac{C_\bullet(U) + C_\bullet(V)}{C_\bullet(V)} \arrow[r,"\iota_\bullet'"] & \displaystyle\frac{C_\bullet(X)}{C_\bullet(V)} \arrow[r] & \displaystyle\frac{C_\bullet(X)}{C_\bullet(U) + C_\bullet(V)} \arrow[r] & 0
		\end{tikzcd}
	\end{equation*}
	\noindent in $\mathsf{Comp}$. Again, there is an associated long exact sequence in homology by proposition \ref{prop:les_homology}, where by the above, every third term is vanishing. Thus $H_n(\iota_\bullet')$ is an isomorphism. Now the second isomorphism theorem together with the fact that $C_\bullet(U \cap V) = C_\bullet(U) \cap C_\bullet(V)$ implies
	\begin{equation*}
		f_\bullet : \frac{C_\bullet(U)}{C_\bullet(U \cap V)} \cong \frac{C_\bullet(U) + C_\bullet(V)}{C_\bullet(V)}.
	\end{equation*}
	Thus for any $n \in \mathbb{Z}$, we have that 
	\begin{equation*}
		H_n(\iota_\bullet' \circ f_\bullet) = H_n(\iota_\bullet') \circ H_n(f_\bullet) : H_n\del[3]{\frac{C_\bullet(U)}{C_\bullet(U \cap V)}} \cong H_n\del[3]{\displaystyle\frac{C_\bullet(X)}{C_\bullet(V)}}.
	\end{equation*}
\end{proof}

\begin{theorem}[The Excision Axiom, Second Formulation]
	Let $X'' \subseteq X' \subseteq X$ be subspaces such that $\wbar{X}'' \subseteq \mathring{X}'$. Then the inclusion $(X\setminus X'',X' \setminus X'') \hookrightarrow (X,X')$ induces an isomorphism
	\begin{equation*}
		H_n(X\setminus X'',X' \setminus X'') \cong H_n(X,X'),
	\end{equation*}
	\noindent for all $n \in \omega$.
\end{theorem}

\subsection*{Reduced Homology}


\subsection*{The Mayer-Vietoris Theorem}

\begin{proposition}[Commutative Braid]
	\label{prop:commutative_braid}
	Consider the diagram
	\begin{equation*}
		\begin{tikzcd}[column sep=1.5ex, row sep=2ex]
			\dots \arrow[dr,green]\arrow[rr,red, bend left] & & E_{n + 1}\arrow[rr,bend left]\arrow[dr,red] & & C_n \arrow[rr,blue,bend left]\arrow[dr,"i_n"] & & F_n \arrow[rr,green,bend left]\arrow[dr,blue] & & B_{n - 1}\arrow[dr,green]\arrow[rr,bend left,red] & & \dots\\
			& D_{n + 1} \arrow[ur,"j_{n + 1}"]\arrow[dr,green] & & A_n \arrow[dr,red]\arrow[ur,blue] & & D_n \arrow[dr,"j_n"]\arrow[ur,green] & & A_{n - 1}\arrow[dr,blue]\arrow[ur,red] & & D_{n - 1} \arrow[dr,green]\arrow[ur,"j_{n - 1}"]\\
			\dots \arrow[ur,"i_{n + 1}"]\arrow[rr,blue,bend right] & & F_{n + 1} \arrow[ur,blue]\arrow[rr,bend right, green] & & B_n \arrow[rr,bend right,red]\arrow[ur,green] & & E_n \arrow[rr,bend right]\arrow[ur,red] & & C_{n - 1}\arrow[rr,bend right, blue]\arrow[ur,"i_{n - 1}"] & & \dots
		\end{tikzcd}
	\end{equation*}
	\noindent in $\mathsf{AbGrp}$, where the blue, green and red strands are long exact sequences. Moreover, assume that either
	\begin{equation*}
		\im i_n \subseteq \ker j_n \qquad \text{or} \qquad \ker j_n \subseteq \im i_n
	\end{equation*}
	\noindent holds for all $n \in \mathbb{Z}$. Then the black strand is also a long exact sequence.
\end{proposition}

\begin{proof}
	\begin{enumerate}[label = \textit{Step \arabic*:}, wide = 0pt]
		\item \textit{Exactness at $C_n$.} 
			\begin{itemize}[wide = 0pt]
				\item $\im(E_{n + 1} \to C_n) \subseteq \ker i_n$. Suppose $e_{n + 1} \mapsto c_n$. Then we have by commutativity that $e_{n + 1} \textcolor{red}{\mapsto} a_n \textcolor{blue}{\mapsto} c_n$. By exactness we have that $a_n \textcolor{red}{\mapsto} 0 \textcolor{green}{\mapsto} 0$ and thus again by commutativity, $a_n \textcolor{blue}{\mapsto} c_n \mapsto 0$. 
				\item $\ker i_n \subseteq \im (E_{n + 1} \to C_n)$. Suppose $c_n \mapsto 0$. Then $c_n \mapsto 0 \textcolor{green}{\mapsto} 0$, hence by commutativity and exactness we find $a_n \textcolor{blue}{\mapsto} c_n$. Again, commutativity yields $a_n \textcolor{red}{\mapsto} b_n \textcolor{green}{\mapsto} 0$. Hence by exactness at $B_n$ we find $f_{n + 1} \textcolor{green}{\mapsto} b_n$ and by commutativity, $f_{n + 1} \textcolor{green}{\mapsto} a_n' \textcolor{red}{\mapsto} b_n$. Thus $a_n - a_n' \in \ker(A_n \textcolor{red}{\to} B_n)$ and thus $e_{n + 1} \textcolor{red}{\mapsto} a_n - a_n'$. By exactness at $A_n$, $e_{n + 1} \textcolor{red}{\mapsto} a_n - a_n' \textcolor{blue}{\mapsto} c_n$ and so by commutativity, $e_{n + 1} \mapsto c_n$. 
			\end{itemize}
		\item \textit{Exactness at $D_n$, assuming $\im i_n \subseteq \ker j_n$.} Left to show is that $\ker j_n \subseteq \im i_n$. Suppose $d_n \mapsto 0$. Then $d_n\mapsto 0 \textcolor{red}{\mapsto} 0$. Hence by commutativity, $d_n \textcolor{green}{\mapsto} f_n \textcolor{blue}{\mapsto} 0$. Thus exactness at $F_n$ implies $c_n \textcolor{blue}{\mapsto} f_n$. By commutativity, $c_n \mapsto d_n' \textcolor{green}{\mapsto} f_n$. By assumption, $d_n' \mapsto 0$. Now $d_n - d_n' \textcolor{green}{\mapsto} 0$ and thus by exactness at $D_n$, we have that $b_n \textcolor{green}{\mapsto} d_n - d_n'$. By commutativity, $b_n \textcolor{red}{\mapsto} 0$ and hence by exactness at $B_n$ we get that $a_n \textcolor{red}{\mapsto} b_n$. By commutativity, $a_n \textcolor{blue}{\mapsto} c_n' \mapsto d_n - d_n'$ and so $c_n + c_n'$ does the job.
		\item \textit{Exactness at $D_n$, assuming $\ker j_n \subseteq \im i_n$.} Left  to show is that $\im i_n \subseteq \ker j_n$. Suppose $c_n \mapsto d_n$. Then $d_n \textcolor{green}{\mapsto} f_n$ and by commutativity we have that $c_n \textcolor{blue}{\mapsto} f_n$. Hence $d_n \textcolor{green}{\mapsto} f_n \textcolor{blue} 0$ by exactness at $F_n$. Hence commutativity yields $d_n \mapsto e_n \textcolor{red}{\mapsto} 0$. By exactness we have that $b_n \textcolor{red}{\mapsto} e_n$. Commutativity implies $b_n \textcolor{green}{\mapsto} d_n' \mapsto e_n$. Hence $d_n - d_n' \mapsto 0$ and thus by assumption $c_n' \mapsto d_n - d_n'$. Now $d_n - d_n' \textcolor{green}{\mapsto} f_n$ and hence by commutativity $c_n' \textcolor{blue}{\mapsto} f_n$. So $c_n - c_' \textcolor{blue}{\mapsto} 0$ and by exactness at $C_n$ we have that $a_n \textcolor{blue}{\mapsto} c_n - c_n'$. By commutativity $a_n \textcolor{red}{\mapsto} b_n' \textcolor{green}{\mapsto} d_n'$. Now $b_n' \textcolor{green}{\mapsto} d_n' \mapsto e_n$. By commutativity $b_n' \textcolor{red}{\mapsto} e_n$, but by exactness at $B_n$, we have that $b_n' \textcolor{red}{\mapsto} 0$. so $e_n = 0$ and thus $d_n \mapsto 0$.
		\item \textit{Exactness at $E_n$.} 
			\begin{itemize}[wide = 0pt]
				\item $\im j_n \subseteq \ker(E_n \to C_{n - 1})$. Suppose $d_n \mapsto e_n$. Consider $d_n \textcolor{green}{\mapsto}f_n \textcolor{blue}{\mapsto} a_{n-1}$. Then by commutativity, $d_n \mapsto e_n \textcolor{red}{\mapsto} a_{n - 1}$. Hence by exactness $e_n \textcolor{red}{\mapsto} a_{n - 1} \textcolor{blue}{\mapsto} 0$ and thus by commutativity, $e_n \mapsto 0$.
				\item $\ker(E_n \to C_{n - 1}) \subseteq \im j_n$. Suppose $e_n \mapsto 0$. Hence by commutativity $e_n \textcolor{red}{\mapsto} a_{n - 1} \textcolor{blue}{\mapsto} 0$. Hence by exactness at $A_{n - 1}$ we get that $f_n \textcolor{blue}{\mapsto} a_{n - 1}$. Again, exactness at $A_{n - 1}$ and commutativity implies that $f_n \textcolor{green}{\mapsto} 0$. Hence by exactness at $F_n$ we have that $d_n \textcolor{green}{\mapsto} f_n$. Commutatvity yields $d_n \mapsto e_n' \textcolor{red}{\mapsto} a_{n - 1}$ and hence $e_n - e_n' \textcolor{red}{\mapsto} 0$. Thus $b_n \textcolor{red}{\mapsto} e_n - e_n'$. Commutativity now implies that $b_n \textcolor{green}{\mapsto} d_n' \mapsto e_n - e_n'$ and hence $d_n + d_n'$ does the job. 
			\end{itemize}
	\end{enumerate}
\end{proof}

\begin{proposition}[Long Exact Sequence of Triples]
	Let $X$ be a topological space and $X'' \subseteq X' \subseteq X$ subspaces. Then there is a long exact sequence
	\begin{equation*}
		\begin{tikzcd}[column sep=4ex]
			\dots \arrow[r] & H_n(X',X'') \arrow[r] & H_n(X,X'') \arrow[r] & H_n(X,X')\arrow[r] & H_{n - 1}(X',X'') \arrow[r] & \dots
		\end{tikzcd}
	\end{equation*}
\end{proposition}

\begin{proof}
	This immediately follows from proposition \ref{prop:commutative_braid} by considering the braid
	\begin{equation*}
		\begin{tikzcd}[column sep=1ex, row sep=2ex]
			\dots \arrow[dr,green]\arrow[rr,red, bend left] & & H_{n + 1}(X,X')\arrow[rr,bend left]\arrow[dr,red] & & H_n(X',X'') \arrow[rr,blue,bend left]\arrow[dr] & & H_{n - 1}(X'')\arrow[rr,green,bend left]\arrow[dr,blue] & & \dots\\
			& H_{n + 1}(X,X'') \arrow[ur]\arrow[dr,green] & & H_n(X') \arrow[dr,red]\arrow[ur,blue] & & H_n(X,X'') \arrow[dr]\arrow[ur,green] & & H_{n - 1}(X')\arrow[dr,blue]\arrow[ur,red]\\
			\dots \arrow[ur]\arrow[rr,blue,bend right] & & H_n(X'') \arrow[ur,blue]\arrow[rr,bend right, green] & & H_n(X) \arrow[rr,bend right,red]\arrow[ur,green] & & H_n(X,X') \arrow[rr,bend right]\arrow[ur,red] & & \dots
		\end{tikzcd}
	\end{equation*}
	\noindent where the red strand is the long exact sequence corresponding to the pair $(X,X')$, the green corresponds to $(X,X'')$ and the blue to $(X,X'')$. Commutativity follows from the fact that $H_n$ is a functor and that the connecting homomorphisms are natural. Hence an application of the commutative braid proposition yields the result.
\end{proof}

\begin{theorem}[Mayer-Vietoris]
	\label{thm:Mayer_Vietoris}
	Let $U$ and $V$ be an open cover for a topological space $X$. Define four inclusions 
	\begin{equation*}
		\iota_U : U \cap V \hookrightarrow U, \iota_V : U \cap V \hookrightarrow V, \jota_U : U \hookrightarrow X, \text{ and } \jota_V : V \hookrightarrow X.
	\end{equation*}
	Then there is a long exact sequence
	\begin{equation*}
		\begin{tikzcd}[column sep=10ex]
			\dots \arrow[r] & H_n(U \cap V) \arrow[r,"{(H_n(\iota_U),H_n(\iota_V))}"] & H_n(U) \oplus H_n(V) \arrow[r,"{H_n(\jota_U)-H_n(\jota_V)}"] & H_n(X) \arrow[r,"D"] & \dots
		\end{tikzcd}
	\end{equation*}
	\noindent where $D : H_n(X) \to H_{n - 1}(U \cap V)$.
\end{theorem}

\begin{definition}[Retract]
	Let $(X,S) \in \ob(\mathsf{Top}^2)$. We say that \bld{$S$ is a retract of $X$},  iff the inclusion $\iota : S \hookrightarrow X$ admits a retraction in $\mathsf{Top}$.
\end{definition}

\begin{definition}[Deformation Retract]
	Let $(X,S) \in \ob(\mathsf{Top}^2)$. We say that \bld{$S$ is a deformation retract of $X$}, iff $S$ is a retract of $X$ such that $\iota \circ r \simeq \id_X$ holds, where $r$ denotes the retraction of the inclusion $\iota : S \hookrightarrow X$.
\end{definition}

\begin{definition}[Strong Deformation Retract]
	Let $(X,S) \in \ob(\mathsf{Top}^2)$. We say that \bld{$S$ is a strong deformation retract of $X$}, iff $S$ is a deformation retract of $X$ such that $\iota \circ r \simeq_S \id_X$.
\end{definition}

\begin{example}
	\label{ex:strong_deformation_retract_spheres}
	Let $n \in \omega$, $n \geq 1$. Define $N := e_{n + 1}$ and $S := -e_{n + 1}$, where $e_{n + 1}$ denotes the $n + 1$-th standard basis vector of $\mathbb{R}^{n + 1}$. Moreover, set
	\begin{equation*}
		U_+ := \mathbb{S}^n \setminus S \qquad \text{and} \qquad U_- := \mathbb{S}^n \setminus N.
	\end{equation*}
	Then $U_+$ and $U_-$ are open subsets of $\mathbb{S}^n$, the extended upper and lower hemisphere, respectively. We claim that \emph{$\mathbb{S}^{n - 1}$ is a strong deformation retract of $U_+ \cap U_-$}, where we include $\mathbb{S}^{n - 1}$ into $U_+ \cap U_-$ via $\iota : \mathbb{S}^{n - 1} \hookrightarrow U_+ \cap U_-$ defined by
	\begin{equation*}
		\iota(x) := (x,0).
	\end{equation*}
	Define $r : U_+ \cap U_- \to \mathbb{S}^{n - 1}$ by setting
	\begin{equation*}
		r(x) := \frac{1}{\sqrt{1 - x_{n + 1}^2}} (x_1,\dots,x_n).
	\end{equation*}
	Then it is easy to see that $r \circ \iota = \id_{\mathbb{S}^{n - 1}}$. Consider $H : (U_+ \cap U_-) \times I \to U_+ \cap U_-$ defined by
	\begin{equation*}
		H(x,t) := \frac{1}{\sqrt{1 + (t^2 - 1)x_{n + 1}^2}}(x_1,\dots,x_n,tx_{n + 1}).
	\end{equation*}
	Then it is also not hard to check that $H : \iota \circ r \simeq_{\mathbb{S}^{n - 1}} \id_{U_+ \cap U_-}$. 
\end{example}

\begin{proposition}[Homology of Spheres]
	For $n \in \omega$, we have
	\begin{equation*}
		\wtilde{H}_k(\mathbb{S}^n) = \ccases{
			\mathbb{Z} & k = n,\\
			0 & k \neq n.
		}
	\end{equation*} 
\end{proposition}

\begin{proof}
	We induct over $n \in \omega$. If $n = 0$, 	
\end{proof}
