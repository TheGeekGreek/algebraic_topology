\chapter{Cellular Homology}
\section*{Cell Complexes}
\subsection*{Adjunction Spaces}

\begin{definition}[Adjunction Space]
	Let $X$ and $Y$ be topologial spaces and let $A \subseteq X$ be a closed subspace. Moreover, let $f \in \mathsf{Top}(A,Y)$. Define the \bld{adjunction space of $X$ and $Y$ along $f$}, written $X \cup_f Y$, to be
	\begin{equation*}
		X \cup_f Y := \textstyle\del[1]{X \coprod Y}/{\sim},
	\end{equation*}
	\noindent where $\sim$ is the smallest equivalence relation on $X \coprod Y$ generated by $a {\sim} f(a)$, for $a \in A$.
\end{definition}

\begin{lemma}
	\label{lem:adjunction_space}
	Let $X$ and $Y$ be topological spaces, $A \subseteq X$ a closed subspace and $f \in \mathsf{Top}(A,Y)$. Then:
	\begin{enumerate}[label = \textup{(}\alph*\textup{)}]
		\item $X \cup_f Y$ with obvious inclusions is the pushout of the diagram
			\begin{equation*}
				\begin{tikzcd}
					A \arrow[d,hook,"\iota"']\arrow[r,"f"] & Y\\
					X
				\end{tikzcd}
			\end{equation*}
			\noindent in $\mathsf{Top}$.
		\item The inclusion $q \circ \iota_Y : Y \to X \cup_f Y$ is a closed embedding.
		\item $q \circ \iota_X\vert_{X \setminus A}$ is an open embedding.
		\item $X \cup_f Y$ is the disjoint union of $(q \circ \iota_X)(X \setminus A)$ and $(q \circ \iota_Y)(Y)$.
	\end{enumerate}
\end{lemma}

\begin{proof}
	To prove (a), simply use that $X \coprod Y$ is a coproduct in $\mathsf{Top}$. Indeed, if we have another cocone for the diagram, we have also a cocone for the coproduct diagram of $X$ and $Y$. Hence there exists a unique continuous map from $X \coprod Y$ to the other vertex, and it is easy to check that this map passes to the quotient.\\
	To prove (b), observe that $q \circ \iota_Y$ with restricted codomain has an obvious inverse defined by $\sbr[0]{y} \mapsto y$. This is well defined since if $y {\sim}y'$, we must have $y = y'$ by definition of the equivalence relation generated by $a{\sim}f(a)$. Let $B \subseteq Y$ closed. Then $q^{-1}\del[1]{q(\iota_Y(B))} = f^{-1}(B) \coprod B$, and thus since $f^{-1}(B)$ is closed in $A$ and $A$ is closed in $X$, $f^{-1}(B)$ is closed in $X$. Hence $f^{-1}(B) \coprod B$ is closed in $X \coprod Y$ by definition of the disjoint union space topology. From this also follows that $q(\iota_Y(Y))$ is closed in $X \cup_f Y$.\\
	Note that since $A$ is closed in $X$, $X \setminus A$ is open in $X$. Similar to part (b), we see that an inverse is given by $\sbr{x} \mapsto x$. Let $U \subseteq X \setminus A$ be open. Then $q^{-1}\del[1]{q(\iota_X(U))} = U$, which is open in $X \setminus A$ and hence in $X$.\\
	
\end{proof}

\begin{definition}[Cells]
	Let $n \in \omega$, $n \geq 1$. Then $\mathbb{E}^n := \mathbb{B}^n \setminus \mathbb{S}^{n-1}$ is called the \bld{standard $n$-cell}. If $X$ is a topological space and $E \subseteq X$ is homeomorphic to $\mathbb{E}^n$, then $E$ is called an \bld{$n$-cell in $X$}. Moreover, if $f \in \mathsf{Top}(\mathbb{S}^{n-1},Y)$, the adjunction space $\mathbb{B}^n \cup_f Y$ is said to be obtained from $Y$ by \bld{attaching an $n$-cell}. 
\end{definition}

\begin{examples}
	~
	\begin{enumerate}[label = \textup{(}\alph*\textup{)},wide = 0pt]
		\item For all $n \geq 1$, $\mathbb{S}^n$ is obtained by attaching an $n$-cell to a point.
		\item $\mathbb{RP}^n = E_0 \cup E_1 \cup \dots E_{n - 1} \cup E_n$.
		\item $\mathbb{CP}^n = E_0 \cup E_2 \cup E_4 \cup \dots \cup E_{2n}$.
		\item For any $m,n \in \omega$, the space $\mathbb{S}^m \times \mathbb{S}^n$ is obtained from $\mathbb{S}^m \vee \mathbb{S}^n$ by attaching a $m + n$-cell.
	\end{enumerate}
\end{examples}

\begin{proposition}
	Let $Y$ be a Hausdorff space, $n \in \omega$, $n > 0$, and $f \in \mathsf{Top}(\mathbb{S}^{n - 1},Y)$. Then if $\iota : Y \hookrightarrow  \mathbb{B}^n \cup_f Y$ denotes inclusion, there is a long exact sequence
	\begin{equation*}
		\begin{tikzcd}
			\dots H_k(\mathbb{S}^{n - 1}) \arrow[r,"H_k(f)"] & H_k(Y) \arrow[r,"H_k(\iota)"] & H_k(\mathbb{B}^n \cup_f Y) \arrow[r] & H_{k - 1}(\mathbb{S}^{n - 1}) \dots
		\end{tikzcd}
	\end{equation*}
\end{proposition}

\begin{proof}
	By lemma \ref{lem:adjunction_space}, we know that $\mathbb{B}^n \cup_f Y = q(\mathbb{E}^n) \dot{\cup} q(Y)$. Hence we can write $\mathbb{B}^n \cup_f Y = U \cup V$, where
	\begin{equation*}
		U := q(\mathbb{B}^n_{1/2}) \qquad \text{and} \qquad V := (\mathbb{B}^n \cup_f Y) \setminus q(0),
	\end{equation*}
	\noindent where $0 \in \mathbb{B}^n$. We claim that $q(Y)$ is a deformation retract of $V$. Indeed, consider $H : V \times I \to V$ defined by
	\begin{equation*}
		H(v,t) := \ccases{
			v & v \in q(Y),\\
			q\del[1]{(1 - t)x + t x/\abs[0]{x}} & v = q(x) \in q(\mathbb{E}^n \setminus \cbr{0}).
		}
	\end{equation*}
	By lemma \ref{lem:adjunction_space} (b), we can identify $q(Y)$ with $Y$. Since $U$ is contractible, Mayer-Vietoris yields a sequence
	\begin{equation*}
		\begin{tikzcd}
			\dots H_k (U \cap V) \arrow[r] & H_k(V) \arrow[r] & H_k{\mathbb{B}^n \cup_f Y} \arrow[r] & H_{k - 1}(U \cap V)\dots
		\end{tikzcd}
	\end{equation*}
	Since $U \cap V$ has the same homotopy type as $\mathbb{S}^{n - 1}$ and $H_k(Y) \cong H_k(V)$, we are almost done. Left to check is that $H_k(\iota_V)$ and $H_k(f)$ can be identified. To this end consider the commutative diagram
	\begin{equation*}
		\begin{tikzcd}
			U \setminus \cbr{0} \arrow[d,hook]\arrow[r] & U \cap V \arrow[d,"\iota_V"]\\
			\mathbb{B}^n \setminus \cbr{0} \arrow[r] & V\\
			\mathbb{S}^{n - 1}\arrow[u,hook] \arrow[r,"f"'] & Y \arrow[u,hook].
		\end{tikzcd}
	\end{equation*}
	We see that all the vertical maps and the top horizontal map induce isomorphisms on homology, thus we have
	\begin{equation*}
		\begin{tikzcd}
			H_k(U \cap V) \arrow[d]\arrow[r,"H_k(\iota_V)"] & H_k(V)\\
			H_k(\mathbb{S}^{n - 1}) \arrow[r,"H_k(f)"'] & H_k(Y) \arrow[u]
		\end{tikzcd}
	\end{equation*}
	\noindent in homology, where the vertical maps are isomorphisms.
\end{proof}

\subsection*{The Relative Homeomorphism Theorem}

\begin{definition}[Strong Deformation Retract]
	Let $X$ be a topological space and $A \subseteq X$ a subspace. We say that \bld{$A$ is a strong deformation retract of $X$}, if $A$ is a deformation retract of $X$ and $\iota \circ r \simeq_A \id_X$.
\end{definition}

\begin{theorem}
	\label{thm:quotient}
	Let $X$ be a topological space and $A \subseteq X$ a closed subspace, such that there exists a neighbourhood $U$ of $A$ in $X$ such that $A$ is a strong deformation retract of $U$. If $q : X \to X/A$ denotes the quotient map and $\ast$ corresponds the equivalence relation of an element in $A$, we have that
	\begin{equation*}
		H_n(q) : H_n(X,A) \to H_n(X/A,\ast)
	\end{equation*}
	\noindent is an isomorphism.
\end{theorem}

\begin{proof}
	There is an obvious commutative diagram comming from the long exact sequences of pairs in homology:
	\begin{equation*}
		\begin{tikzcd}
			\dots H_n(A) \arrow[r]\arrow[d] & H_n(X) \arrow[r]\arrow[d,"\id"] & H_n(X,A) \arrow[r]\arrow[d] & H_{n - 1}(A) \arrow[r]\arrow[d] & H_{n - 1}(X)\arrow[d,"\id"]\dots\\
			\dots H_n(U) \arrow[r] & H_n(X) \arrow[r] & H_n(X,U) \arrow[r] & H_{n - 1}(U) \arrow[r] & H_{n - 1}(X)\dots
		\end{tikzcd}
	\end{equation*}
	Since all vertical maps except the middle one are isomorphisms, the five lemma yields that the middle one is also an isomorphism. Now $\cbr{\ast}$ is a strong deformation retract of $U/A$ in $X/A$. Indeed, if $H : U \times I \to U$ is the map corresponding to the strong deformation retract of $A$ in $U$, we see that $H : U \times I \to U/A$ passes to the quotient to yield a map $\wtilde{H} : U/A \times I \to U/A$ with the desired properties. Hence applying the above argument to the induced case, yields
	\begin{equation*}
		H_n(X/A,\ast) \cong H_n(X/A,U/A).
	\end{equation*}
	Consider
	\begin{equation*}
		\begin{tikzcd}
			H_n(X,A)\arrow[d,"H_n(q)"']\arrow[r,leftrightarrow,"\cong"] & H_n(X,U)\arrow[r,leftrightarrow,"\cong"] & H_n(X\setminus A,U \setminus A) \arrow[d,"H_n(q)"]\\
			H_n(X/A,\ast) \arrow[r,leftrightarrow,"\cong"'] & H_n(X/A,U/A)\arrow[r,leftrightarrow,"\cong"'] & H_n\del[1]{(X/A) \setminus \cbr{\ast},(U/A)\setminus \cbr{\ast}}.
		\end{tikzcd}
	\end{equation*}
	Now the right-hand side $H_n(q)$ is an isomorphism since $q$ is a homeomorphism restricted to $X\setminus A$.
\end{proof}

\begin{definition}[Relative Homeomorphism]
	Let $f \in \mathsf{Top}^2 \del[1]{(X,A),(Y,B)}$. We say that $f$ is a \bld{relative homeomorphism}, if $f : X \setminus A \to Y \setminus B$ is a homeomorphism.
\end{definition}

\begin{theorem}[The Relative Homeomorphism Theorem]
	Let $f \in \mathsf{Top}^2 \del[1]{(X,A),(Y,B)}$ be a relative homeomorphism between a compact space $X$ and a compact Hausdorff space $Y$ and where $A$ and $B$ are closed subspaces. Moreover, assume that there exist neighbourhoods $U$ and $V$, such that $A$ and $B$ are strong deformation retracts of $U$ and $V$, respectively. Then
	\begin{equation*}
		H_n(f) : H_n(X,A) \to H_n(Y,B)
	\end{equation*}
	\noindent is an isomorphism for all $n \in \omega$.
\end{theorem}

\begin{proof}
	The assumptions imply the existence of a well-defined continuous bijective map $\wtilde{f}$, such that the following diagram commutes:
	\begin{equation*}
		\begin{tikzcd}
			(X,A) \arrow[d]\arrow[r,"f"] & (Y,B)\arrow[d]\\
			(X/A,\ast) \arrow[r,"\wtilde{f}"'] & (Y/B,\ast).
		\end{tikzcd}
	\end{equation*}
	Since every quotient of a compact space is compact (see \cite[96]{lee:topological_manifolds:2011}), we get that $X/A$ is compact. Moreover, by \cite[102]{lee:topological_manifolds:2011}, we have that $Y/B$ is Hausdorff. Hence the closed map lemma implies that $\wtilde{f}$ is a homeomorphism. Applying homology yields
	\begin{equation*}
		\begin{tikzcd}
			H_n(X,A) \arrow[d]\arrow[r,"H_n(f)"] & H_n(Y,B)\arrow[d]\\
			H_n(X/A,\ast) \arrow[r,"H_n(\wtilde{f})"'] & H_n(Y/B,\ast),
		\end{tikzcd}
	\end{equation*}
	\noindent where the two vertical maps are isomorphisms by theorem \ref{thm:quotient}.	
\end{proof}

\section*{The Degree}
\begin{definition}[Degree]
	Let $n \geq 1$ and $f : \mathbb{S}^n \to \mathbb{S}^n$ continuous. Then since $H_n(\mathbb{S}^n)$ is infinite cyclic, there is a unique integer $\deg f$ such that $H_n(f)$ is the multiplication by $\deg f$. This integer is called the \bld{degree of $f$}.
\end{definition}

\begin{proposition}
	Let $n \geq 1$ and $f,g \in \mathsf{Top}(\mathbb{S}^n,\mathbb{S}^n)$. Then:
	\begin{enumerate}[label = \textup{(}\alph*\textup{)}]
		\item $\deg(g \circ f) = \deg g \deg f$.
		\item $\deg \id_{\mathbb{S}^n} = 1$.
		\item If $f$ is constant, then $\deg f = 0$.
		\item If $f \simeq g$, then $\deg f = \deg g$.
		\item If $f$ is a homotopy equivalence, then $\deg f = \pm 1$.
	\end{enumerate}
\end{proposition}

\begin{proposition}
	Let $n \geq 1$ and $A \in \mathrm{O}(n + 1)$. Set $f := A \vert_{\mathbb{S}^n}$. Then $\deg f = \det A$.
\end{proposition}

\begin{proof}
	The group $\mathrm{O}(n + 1)$ has two connected components distinguished by $\det$. By homotopy invariance it is enough to check the result for one $A$ in each component. Since the identity matrix induces degree $1$, we have to check it only for the other component. We take $A$ the reflection in some hyperplane in $\mathbb{R}^{n + 1}$. Divide $\mathbb{S}^n$ in two hemispheres which are preserved by $A$. Then $f$ induces a reflection $f'$ in the equatorial $\mathbb{S}^{n - 1}$. Using Mayer-Vietorsi we obtain the commutative diagram
	\begin{equation*}
		\begin{tikzcd}
			H_n(\mathbb{S}^n) \arrow[d,"H_n(f)"']\arrow[r,leftrightarrow,"\cong"] & H_{n - 1}(\mathbb{S}^{n - 1})\arrow[d,"H_{n - 1}(f')"]\\
			H_n(\mathbb{S}^n) \arrow[r,leftrightarrow,"\cong"'] & H_{n - 1}(\mathbb{S}^{n - 1}).
		\end{tikzcd}
	\end{equation*}
	Hence $\deg f = \deg f'$. Thus it suffices to prove the result for $n = 1$.
\end{proof}

\begin{corollary}
	Let $n \geq 1$. Then the antipodal map $\alpha : \mathbb{S}^n \to \mathbb{S}^n$, $\alpha(x) := -x$, has degree $(-1)^{n + 1}$.
\end{corollary}

\begin{theorem}[The Hairy Ball Theorem]
	There exists a nowhere vanishing vector field on $\mathbb{S}^n$ if and only if $n$ is odd.
\end{theorem}

\begin{proof}
	Let $V$ be a nowhere vanishing vector field on $\mathbb{S}^n$. Define $H : \mathbb{S}^n \times I \to \mathbb{S}^n$ by
	\begin{equation*}
		H(x,t) := (\cos \pi t) x + (\sin \pi t) V(x)/\abs[0]{V(x)}.
	\end{equation*}
	It is easy to check that $H : \id_{\mathbb{S}^n} \simeq \alpha$, but then we have that 
	\begin{equation*}
		(-1)^{n + 1} = \deg \alpha = \deg \id_{\mathbb{S}^n} = 1,
	\end{equation*}
	\noindent which is only possible if $n$ is odd. Conversly, if $n = 2m - 1$, define $V : \mathbb{R}^{2m} \to \mathbb{R}^{2m}$ by
	\begin{equation*}
		V(x_1,y_1,\dots,x_m,y_m) := (-y_1,x_1,\dots,-y_m,x_m).
	\end{equation*}
\end{proof}

\begin{theorem}
	An odd map has odd degree.
\end{theorem}

\section*{Cellular Homology}

\begin{definition}[Cell-Like Filtration]
	Let $X$ be a topological space. A \bld{cell-like filtration $\mathcal{F}$ of $X$} is an increasing sequence $\mathcal{F}$, $\varnothing =: F^{-1} \subseteq F^0 \subseteq F^1 \subseteq \dots$ of weakly Hausdorff closed subspaces such that $X = \bigcup_{n \in \omega} F^n$ and
	\begin{equation*}
		H_k(F^n,F^{n - 1}) = 0,
	\end{equation*}
	\noindent for all $k \neq n$.
\end{definition}

\begin{proposition}
	Let $X$ be a topological space with cell-like filtration $\mathcal{F}$. Then:
	\begin{enumerate}[label = \textup{(} \alph*\textup{)}]
		\item $H_k(F^n) = 0$ for $k > n$.
		\item The inclusion $F^n \hookrightarrow X$ induces isomorphisms $H_k(F^n) \cong H_k(X)$ for $k < n$.
	\end{enumerate}
\end{proposition}

\begin{proof}
	For proving (a), we consider the long exact sequence associated to the pair $(F^n,F^{n-1})$:
	\begin{equation*}
		\begin{tikzcd}
			\dots H_{k + 1}(F^n,F^{n - 1}) \arrow[r] & H_k(F^{n-1}) \arrow[r] & H_k(F^n) \arrow[r] & H_k(F^n,F^{n-1})\dots
		\end{tikzcd}
	\end{equation*}
	If $k \neq n,n - 1$, both end terms vanish by definition of a cell-like filtration. Hence we get that
	\begin{equation*}
		H_k(F^{n - 1}) \cong H_k(F^n)
	\end{equation*}
	\noindent for all $k \neq n,n - 1$. In particular, if $k > n$ we get a chain of isomorphisms
	\begin{equation*}
		H_k(F^n) \cong H_k(F^{n-1}) \cong \dots \cong H_k(F^0) = H_k(F^0,\varnothing) = H_k(F^0,F^{-1}) = 0.
	\end{equation*}
	Also we have that $H_k(F^{k + m}) \cong H_k(F^{k + 1})$ for all $m \geq 1$. Hence
	\begin{equation*}
		\textstyle H_k(F^{k + 1}) \cong \colim_{n \geq k + 1} H_k(F^n) = H_k(X).
	\end{equation*}
\end{proof}

\begin{theorem}
	Let $X$ be a topological space with cell-like filtration $\mathcal{F}$. Then
	\begin{equation*}
		H_n(X) \cong H_n(X,\mathcal{F})
	\end{equation*}
	\noindent for all $n \geq 0$.
\end{theorem}

\begin{proof}
	Consider
	\begin{equation*}
		\begin{tikzcd}
			& H_{n + 1}(F^{n + 1},F^n)\arrow[d,"\delta_{n + 1}"']\arrow[dr,"\partial^\mathcal{F}_{n + 1}"] & & 0\arrow[d]\\
			0 \arrow[r] & H_n(F^n)\arrow[d] \arrow[r,"\eta_n"] & H_n(F^n,F^{n - 1})\arrow[dr,"\partial^\mathcal{F}_n"'] \arrow[r,"\delta_n"] & H_{n - 1}(F^{n - 1})\arrow[d,"\eta_{n - 1}"]\\
			& H_n(F^{n + 1})\arrow[d] & & H_{n - 1}(F^{n - 1},F^{n - 2})\\
			& 0.
		\end{tikzcd}
	\end{equation*}
	Then we have 
	\begin{align*}
		H_n(X) &\cong H_n(F^{n + 1})\\
		&\cong H_n(F^n)/\im \delta_{n + 1}\\
		&\cong \im \eta_n/\im \eta_n \circ \delta_{n + 1}\\
		&\cong \ker \delta_n/\im \partial^\mathcal{F}_{n + 1}\\
		&\cong \ker \eta_{n - 1}\circ \delta_n/ \im \partial^\mathcal{F}_{n + 1}\\
		&\cong \ker \partial^\mathcal{F}_n/\im \partial^\mathcal{F}_{n + 1}\\
		&= H_n(X,\mathcal{F}).
	\end{align*}
\end{proof}

\begin{definition}
	Let $Y,Z \in \ob(\mathsf{Top})$ and $n \geq 1$. We say that \bld{$Z$ is obtained from $Y$ by attaching $n$-cells}, if there exists a pushout
	\begin{equation*}
		\begin{tikzcd}
			\coprod_{\alpha \in A} \mathbb{S}_\alpha^{n - 1} \arrow[d,hook]\arrow[r,"f"] & Y \arrow[d]\\
			\coprod_{\alpha \in A} \mathbb{B}^n_\alpha \arrow[r,"g"'] & Z.
		\end{tikzcd}
	\end{equation*}
	Moreover, for all $\alpha \in A$ set $f_\alpha := f\vert_{\mathbb{S}^{n - 1}_\alpha}$ and $g_\alpha := g\vert_{\mathbb{B}^n_\alpha}$. We call $g_\alpha$ the \bld{characteristic map} of the $n$-cell $g(\mathbb{E}^n_\alpha)$ and $f_\alpha$ its \bld{attaching map}.
\end{definition}

\begin{definition}
	Let $X$ be a cell-complex and $n \geq 1$. Moreover, suppose $e_\lambda$ is an $n$-cell and $e_\nu$ is an $n - 1$-cell. Define $h_{\lambda,\nu} : \mathbb{S}^{n - 1}_\lambda \to \mathbb{S}_\nu^{n - 1}$, where $\mathbb{S}_\nu^{n - 1} := g_\nu(\mathbb{B}^{n - 1}_\nu)/f_\nu(\partial \mathbb{B}^{n - 1}_\nu)$, to be the composition
	\begin{equation*}
		\begin{tikzcd}
			\mathbb{S}_\lambda^{n - 1} \arrow[r,"f_\lambda"] & X^{n - 1} \arrow[r] & X^{n - 1}/X^{n - 1} \arrow[r,"q_\nu"] & \mathbb{S}^{n - 1}_\nu.
		\end{tikzcd}
	\end{equation*}
	Also, define
	\begin{equation*}
		\sbr[0]{e_\lambda : e_\nu} := \deg(h_{\lambda,\nu}).
	\end{equation*}
\end{definition}

\begin{theorem}[Cellular Boundary Formula]
	Let $X$ be a cell-complex. Then the cellular boundary operator $\partial^{\mathrm{cell}} : C_n^{\mathrm{cell}}(X) \to C_{n - 1}^{\mathrm{cell}}(X)$ is given by
	\begin{equation*}
		\partial^{\mathrm{cell}}e_\lambda = \sum_\nu \sbr[0]{e_\lambda : e_\nu} e_\nu,
	\end{equation*}
	\noindent on generators.
\end{theorem}

\begin{proof}
	
\end{proof}<++>
